\documentclass{article}

\usepackage{../preamble}
\usepackage{../macros}

\title{Notes for High-Dimensional Probability:\\ Random Vectors in High Dimensions}

\author{Isak Falk}

\begin{document}

\maketitle

\section{Preliminaries}

\begin{proposition}
  \label{prop:sub-gauss-sub-exp-norm-relation}
  Let \(X\) be a real-values random variable and \(Y = \sqrt{\abs{X}}\). Then
  \(X\) is sub-exponential if and only if \(Y\) is sub-Gaussian, and in such
  case \(\norm{X}_{\psi_{1}} = \norm{Y}_{\psi_{2}}^{2}\)
\end{proposition}

\begin{proposition}
  \label{prop:centering-of-sub-exp-rvs}
  Let \(X\) be a real-valued sub-exponential variable, then the centered random
  variable \(X - \E X\) is sub-exponential and
  \begin{equation}
    \norm{X - \E X}_{\psi_{1}} \leq (1 + \frac{2}{\ln 2}) \norm{X}_{\psi_{2}}.
  \end{equation}
\end{proposition}

\begin{proof}
  The proof is analogous to the sub-Gaussian case but with different constant
  since the definition of the norm is different,
  \begin{equation}
    \norm{X - \E X}_{\psi_{1}} \leq \norm{X}_{\psi_{1}} + \norm{\E X}_{\psi_{1}},
  \end{equation}
  and
  \begin{equation}
    \norm{\E X}_{\psi_{1}} = \frac{\abs{\E X}}{\ln 2} \leq \frac{\E \abs{X}}{\ln 2} = \frac{\norm{X}_{1}}{\ln 2} \leq \frac{2}{\ln 2}\norm{X}_{\psi_{1}},
  \end{equation}
  where we have used the definition of sub-exponential norm for constant
  functions, Jensen and bound of \(L^{p}\) norm of \(X\) by sub-exponential
  norm. We thus have that
  \(\norm{X - \E X}_{\psi_{1}} \leq (1 + \frac{2}{\ln 2})\norm{X}_{\psi_{1}}\)
  which is what we wanted to show.

\end{proof}

\begin{theorem}[Bernstein's theorem]
  \label{thm:bernsteins-ineq}
  Let \((X_{i})_{i=1}^{n}\) be a sequence of independent real-valued zero-mean
  random variables such that \(\norm{X_{i}}_{\psi_{1}} < \infty\). Then, for
  every \(t > 0\)
  \begin{equation}
    \Pr(\abs*{\frac{1}{n}\sum_{i=1}^{n}X_{i}} > t) \leq 2\exp(-cn\min(\frac{t^{2}}{K^{2}}, \frac{t}{K})),
  \end{equation}
  where \(K = \max_{i}\norm{X_{i}}_{\psi_{1}}\) and \(c > 0\) is some absolute constant.
\end{theorem}

\section{Concentration of the Norm}
\begin{theorem}[Concentration of the \(L_{2}\) norm]
Let \(X = (X_{1}, \dots, X_{n}) \in \R^{n}\) be a random vector with independent
sub-gaussian coordinates \(X_{i}\) that satisfy \(\E X_{i}^{2} = 1\). Then
\begin{equation}
  \norm{\norm{X}_{2} - \sqrt{n}}_{\psi_{2}} \leq CK^{2},
\end{equation}
where \(K = \max_{i} \norm{X_{i}}_{\psi_{2}}\) and \(C\) is an absolute constant.
\end{theorem}

\begin{proof}
  We first note that \(K \geq 1\). By Jensen's Inequality we have that
  \(\E \exp(\frac{X_{i}^{2}}{t^{2}}) \geq \exp(\frac{\E X_{i}^{2}}{t^{2}}) = \exp(t^{-2})\)
  and using \(t = 1\) we see that \(\E \exp(X_{i}^{2}) \geq e > 2\) so
  \(\norm{X_{i}}_{\psi_{2}} \geq 1\) for all \(i \in \{1, \dots, n\}\). Since
  \(K = \max_{i}\norm{X_{i}}_{\psi_{2}} \geq 1\) we are done.

  Now consider the quantity \(\frac{1}{n}\norm{X}_{2}^{2} - 1\) which we can write as
  \begin{equation}
    \frac{1}{n}\norm{X}_{2}^{2} - 1 = \frac{1}{n}\sum_{i=1}^{n}(X_{i}^{2} - 1) = \frac{1}{n}\sum_{i=1}^{n}Y_{i},
  \end{equation}
  where \(Y_{i} = X_{i}^{2} - 1\). Since \(\E X_{i}^{2} = 1\) for any \(i\),
  \((Y_{i})_{i=1}^{n}\) is a vector of zero-centred random variables. Since
  \(\norm{X_{i}}_{\psi_{2}} < \infty\) we can show that
  \(\norm{Y_{i}}_{\psi_{1}} < \infty\) since
  \begin{align}
    \label{al:1}
    \norm{X_i^2 - 1}_{\psi_1} & \leq C\norm{X_i^2}_{\psi_1} \\
                              & = C\norm{X_i}_{\psi_2}^2 \\
                              & \leq C K^2,
  \end{align}
  using centring property of sub-exponentials
  \cref{prop:centering-of-sub-exp-rvs} (noting that \(C > 1\)),
  \cref{prop:sub-gauss-sub-exp-norm-relation} and definition of \(K\). Since
  \((Y_{i})_{i=1}^{n}\) are independent, zero-mean and by the above calculation sub-exponential, we can apply Bernstein's Inequality
  \cref{thm:bernsteins-ineq} which for any \(u \geq 0\) means that
  \begin{equation}
    \Pr(\abs*{\frac{1}{n}\sum_{i=1}^{n}Y_{i}} > u) \leq 2 \exp(-cn\min(\frac{u^{2}}{C^{2}K^{4}}, \frac{u}{CK^{2}}))
  \end{equation}
  for some \(c > 0\). Since \(C, K > 1\) we have that \(C^{2} > C\) and
  \(K^{4} > K^{2}\), so we can write
  \begin{equation}
    \label{eq:conc-norm-bernstein}
    \exp(-cn\min(\frac{u^{2}}{C^{2}K^{4}}, \frac{u}{CK^{2}})) \leq \exp(-\frac{cn}{C^{4}K^{4}}\min(u^{2}, u)).
  \end{equation}

  So far we have proved a concentration bound on \(\norm{X}_{2}^{2}\). We will
  finish the proof by relating \(\norm{X}_{2}\) to \(\norm{X}_{2}^{2}\). Note
  the following; for any \(z, \delta \geq 0\)
  \begin{equation}
    \label{eq:zp1-delta-observation}
    \abs{z - 1} \geq \delta \Rightarrow \abs{z^{2} - 1} \geq \max(\delta, \delta^{2}).
  \end{equation}
  We show it by first noting that
  \(\abs{z^{2} - 1} = \abs{z - 1}\abs{z + 1} \geq \abs{z + 1}\delta\), and see
  that if \(\delta \in [0, 1)\) then \(\abs{z + 1} \geq \delta\) and if
  \(\delta \geq 1\) then since \(\abs{z + 1} \geq 1\) we have that
  \(\abs{z^{2} + 1} \geq \delta\). We can write this compactly as
  \cref{eq:zp1-delta-observation}.

  Now, consider any \(\delta \geq 0\) and the expression
  \(\abs{\frac{1}{\sqrt{n}}\norm{X}_{2} - 1} \geq \delta\). Using
  \cref{eq:zp1-delta-observation} with \(z = \frac{1}{\sqrt{n}}\norm{X}_{2}\) we
  see that
  \begin{equation}
    \label{eq:4}
    \abs{\frac{1}{\sqrt{n}}\norm{X}_{2} - 1} \geq \delta \Rightarrow \abs{\frac{1}{n}\norm{X}^{2}_{2} - 1} \geq \max(\delta, \delta^{2}).
  \end{equation}
  In terms of events, this means that
  \begin{equation}
    \Pr(\abs*{\frac{1}{\sqrt{n}}\norm{X}_{2} - 1} \geq \delta) \leq \Pr(\abs*{\frac{1}{n}\norm{X}^{2}_{2} - 1} \geq \max(\delta, \delta^{2})) \leq 2 \exp(-\frac{cn}{C^{4}K^{4}}\delta^{2}),
  \end{equation}
  where in the final inequality we have used \cref{eq:conc-norm-bernstein}
  together with
  \begin{equation}
    \delta^{2} = \min(\max(\delta, \delta^{2}), \max(\delta, \delta^{2})^{2}).
  \end{equation}
  Letting \(t = \delta \sqrt{n}\) we obtain the bound
  \begin{equation}
    \Pr(\abs{\norm{X}_{2} - \sqrt{n}} \geq t) \leq 2\exp(-\frac{ct^{2}}{C^{4}K^{4}}),
  \end{equation}
  for any \(t \geq 0\) and this is equal to the conclusion of the theorem.
\end{proof}

\begin{remark}
  The above bound tells us that with high probability, \(X\) takes values very
  close to the sphere of radius \(\sqrt{n}\). In particular, for a fixed probability \(X\) stays
  within a constant distance from that sphere independently of the dimension
  \(n\). This is due to the fact that \(\norm{X}_{2}^{2}\) has mean \(n\) and
  standard deviation \(O(\sqrt{n})\) since
  \begin{equation}
    \V(\norm{X}_{2}^{2}) = \sum_{i=1}\V(X_{i}^{2}) = n \V(X_{1}^{2}),
  \end{equation}
  due to independence of the coordinates of \(X\), and so
  \(\sqrt{\V(\norm{X}_{2}^{2})} = \sqrt{n} \cdot \mathrm{std}(X_{1}^{2})\).
  \(\sqrt{n \pm O(\sqrt{n})} = \sqrt{n} \pm O(1)\) since by Taylor expansion
  around \(\sqrt{n}\) on the interval \([n - c \sqrt{n}, n + c \sqrt{n}]\) where
  \(c\) need to be chosen so that \(n - c \sqrt{n} \geq 0\) we see that
  \begin{align}
    \sqrt{n \pm c\sqrt{n}} = \sqrt{n} + R_{1}(c\sqrt{n})
  \end{align}
  where \(R_{1}(x) = \) [ Need to fill in ].
\end{remark}

\section{Covariance, second moments, whitening and isotropy}
\begin{definition}
  The covariance matrix of a random vector \(X \in \R^{n}\) is
  \begin{equation}
    \Cov(X) = \E(X - \E X)(X - \E X)\tran = \E X X\tran - \E X (\E X)\tran.
  \end{equation}
\end{definition}

\begin{definition}
  The second moment matrix of a random vector \(X \in \R^{n}\) is
  \begin{equation}
  \Sigma = \Sigma(X) = \E X X\tran.
  \end{equation}
\end{definition}

Note that if \(\E X = 0\), then \(\Cov(X) = \Sigma(X)\). So we can mostly
consider the second moment matrix without loss of generality in place of the
covariance as long as we center our variable \(X\). Both of the matrices are
symmetric and positive semi-definite.

\begin{definition}
  A random vector \(X \in \R^{n}\) is called \emph{isotropic} if
  \begin{equation}
    \Sigma(X) = \E X X\tran = I.
  \end{equation}
\end{definition}

For a random variable \(X \in \R^{n}\) with covariance matrix \(\Sigma\) and
mean \(\mu\) we can always put it into a isotropic form by whitening
\(Z = \Sigma^{-1/2}(X - \mu)\). Similarly, the transformation using a psd matrix
\(\Sigma\),
\(X = \mu + \Sigma^{1/2}Z\) of a mean-zero isotropic random vector
\(Z \in \R^{n}\) leads to a random vector with mean \(\mu\) and covariance
\(\Sigma\). This observation means that in many cases we may focus on random
isotropic mean-zero random vectors without loss of generality.

\begin{lemma}
  \label{lem:isotropy-marginal-characterisation}
  A random vector \(X \in \R^{n}\) is isotropic if and only if
  \begin{equation}
    \E \scal{X}{u} = \norm{u}^{2}
  \end{equation}
  for all \(u \in \R^{n}\). Equivalently we could have used
  \begin{equation}
    \E \scal{X}{u} = 1
  \end{equation}
  for all \(u \in S^{n-1}\), the \(n\)-dimensional unit sphere.
\end{lemma}

\begin{proof}
  Two symmetric matrices \(A, B\) are equal if and only if
  \(x\tran A x = x\tran B\) for any \(x \in \R^{n}\). Thus \(X\) is isotropic if
  and only if
  \begin{equation}
    u\tran \E X X\tran u = \E \scal{X}{u} = u\tran I u = \norm{u}_{2}^{2}
  \end{equation}
  for all \(u \in \R^{n}\) which is what we wanted to show.
\end{proof}

\begin{lemma}
  Let \(X\) be an isotropic random vector in \(\R^{n}\). The
  \begin{equation}
    \E \norm{X}^{2}_{2} = n.
  \end{equation}
  Moreover, if \(X, Y\) are two independent isotropic random vectors in
  \(\R^{n}\), then
  \begin{equation}
    \E\scal{X}{Y}^{2} = n
  \end{equation}
\end{lemma}

\begin{proof}
  First we have
  \begin{align}
    \E\norm{X}_{2}^{2} & = \E X\tran X \\
                       & = \E \Tr(X X\tran) \\
                       & = \Tr(\E X X\tran) \\
                       & = \Tr(I) \\
                       & = n.
  \end{align}
  For the second part we use the so called \emph{law of total expectation} to
  write
  \begin{equation}
    \E \scal{X}{Y} = \E_{X} \E_{Y}(\scal{x}{Y} | X = x) = \E_{X} \norm{X}_{2}^{2} = n
  \end{equation}
  by the \cref{lem:isotropy-marginal-characterisation} and reusing the first
  part of the proof.
\end{proof}

\begin{corollary}
  Let \(X, Y\) be independent mean-zero isotropic random vectors in \(R^{n}\),
  then
  \begin{equation}
    \E \norm{X - Y}_{2}^{2} = 2n
  \end{equation}
\end{corollary}

\begin{proof}
  \begin{align}
    \E \norm{X - Y}_{2}^{2} & = \E \norm{X}_{2}^{2} + \E \norm{Y}_{2}^{2} - 2 \E \scal{X}{Y} \\
                            & = 2n - 0 \\
                            & = 2n
  \end{align}
\end{proof}

\section{Examples of high-dimensional distributions}
Let \(X \sim \Unif(\sqrt{n} S^{n-1})\) mean that the law of \(X\) is the uniform
measure on the zero-centered sphere of radius \(\sqrt{n}\) in \(n\) dimensions.

\begin{theorem}[Isotropy of uniform random variable on \(\sqrt{n} S^{n-1}\)]

\end{theorem}

\end{document}
