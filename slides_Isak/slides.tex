\documentclass{beamer}
\usepackage{../preamble_slides}
\usepackage{../macros}

\usetheme{metropolis}

%%% Remove nav symbols (and shift any logo down to corner)
\setbeamertemplate{navigation symbols}{\vspace{-2ex}}

\title{Random Vectors in High Dimensions}
\author{Isak Falk}
\institute{IIT}

\begin{document}
  \maketitle
  \section{Concentration of norm}
  \begin{frame}{Concentration of the \(\ell^{2}\) Norm}
      Let \(X = (X_{1}, \dots, X_{n}) \in \R^{n}\) be a random vector with independent
      sub-gaussian coordinates \(X_{i}\) that satisfy \(\E X_{i}^{2} = 1\). Then
      \begin{equation}
        \norm{\norm{X}_{2} - \sqrt{n}}_{\psi_{2}} \leq CK^{2},
      \end{equation}
      where \(K = \max_{i} \norm{X_{i}}_{\psi_{2}}\) and \(C\) is an absolute
      constant.
  \end{frame}

  \begin{frame}{Proof (Concentration of the \(\ell^{2}\) Norm)}
    Proof strategy\pause
    \begin{itemize}
      \item Show that \(\frac{1}{n}\norm{X}_{2}^{2} - 1 = \sum_{i=1}Y_{i} \)
        where \(Y_{i} = X_{i}^{2} - 1\) are sub-exponential random variables\pause
      \item Use Bernstein's theorem to show that \(\norm{X}_{2}^{2}\)
        concentrates around \(n\)\pause
      \item Show that
        \(\abs{\frac{1}{\sqrt{n}}\norm{X}_{2} - 1} \geq \delta \Rightarrow \abs{\frac{1}{n}\norm{X}_{2}^{2} - 1} \geq \max(\delta, \delta^{2})\)\pause
      \item Combine to show
        \begin{equation}
          \Pr(\abs{\norm{X}_{2} - \sqrt{n}} \geq t) \leq 2 \exp(-\frac{ct^{2}}{C^{4}K^{4}})
        \end{equation}
    \end{itemize}
  \end{frame}

  \begin{frame}{Proof (Concentration of the \(\ell^{2}\) Norm)}
    We first note that \(K \geq 1\)\pause
    \begin{itemize}
      \item By definition
        \(\norm{X_{i}}_{\psi_{2}} = \inf\{t > 0 \: | \: \E\exp(X_{i}^{2} / t^{2}) \leq 2\}\)\pause
      \item
        By Jensen's inequality
        \begin{equation}
          \E \exp(X_{i}^{2} / t^{2}) \geq \exp(\E X_{i}^{2} / t^{2}) = \exp(t^{-2})
        \end{equation}
        since \(\E X_{i}^{2} = 1\) by assumption\pause
      \item Thus \(\norm{X_{i}}_{\psi_{2}}\) satisfies
        \(\exp(\norm{X_{i}}_{\psi_{2}}^{-2}) \leq 2\) \pause
      \item Isolating \(\norm{X_{i}}_{\psi_2}\) shows that
        \(\norm{X_{i}}_{\psi_{2}} \geq 1 / \sqrt{\ln 2} \geq 1\)\pause
      \item Thus \(K = \max_{i}\norm{X_{i}}_{\psi_{2}} \geq 1\)
    \end{itemize}
  \end{frame}

  \begin{frame}{Proof (Concentration of the \(\ell^{2}\) Norm)}
    \begin{proposition}[Centering of sub-exponentials]
      \label{prop:centering-of-sub-exp-rvs}
      Let \(Z\) be a real-valued sub-exponential variable, then the centered random
      variable \(Z - \E Z\) is sub-exponential and
      \begin{equation}
        \norm{Z - \E Z}_{\psi_{1}} \leq (1 + \frac{2}{\ln 2}) \norm{Z}_{\psi_{1}}.
      \end{equation}
    \end{proposition}
    
    \begin{proposition}
      \label{prop:sub-gauss-sub-exp-norm-relation}
      Let \(X\) be a real-values random variable and \(Y = \sqrt{\abs{X}}\). Then
      \(X\) is sub-exponential if and only if \(Y\) is sub-Gaussian, and in such
      case \(\norm{X}_{\psi_{1}} = \norm{Y}_{\psi_{2}}^{2}\).
    \end{proposition}
  \end{frame}

  \begin{frame}{Proof (Concentration of the \(\ell^{2}\) Norm)}
    Let \(Y_{i} = X_{i}^{2} - \E X_{i}^{2} = X_{i}^{2} - 1\). \pause
    \((Y_{i})_{i}^{n}\) is a collection of
    zero-mean, sub-exponential independent random variables. \pause By previous
    slide
    \begin{equation}
      \norm{Y_{i}}_{\psi_{1}} \leq (1 + 2 / \ln 2)\norm{X_{i}^{2}}_{\psi_{1}} = (1 + 2 / \ln 2)\norm{X_{i}}_{\psi_{2}}^{2} \leq C K^{2},
    \end{equation}
    where \(C = (1 + 2 / \ln 2) \geq 1\).
  \end{frame}

  \begin{frame}{Proof (Concentration of the \(\ell^{2}\) Norm)}
    \begin{theorem}[Bernstein]
      \label{thm:bernsteins-ineq}
      Let \((Z_{i})_{i=1}^{n}\) be a sequence of independent, real-valued, zero-mean
      random variables such that \(\norm{Z_{i}}_{\psi_{1}} < \infty\). Then, for
      every \(t > 0\)
      \begin{equation}
        \Pr(\abs*{\frac{1}{n}\sum_{i=1}^{n}Z_{i}} > t) \leq 2\exp(-cn\min(\frac{t^{2}}{A^{2}}, \frac{t}{A})),
      \end{equation}
      where \(A = \max_{i}\norm{Z_{i}}_{\psi_{1}}\) and \(c > 0\) is some absolute constant.
    \end{theorem}
  \end{frame}

  \begin{frame}{Proof (Concentration of the \(\ell^{2}\) Norm)}
    \((Y_{i})_{i=1}^{n}\) satisfies the condition of Bernstein's theorem, \pause for any \(u > 0\),
    \begin{equation}
      \Pr(\abs*{\frac{1}{n}\sum_{i=1}^{n}Y_{i}} > u) \leq 2 \exp(-cn\min(\frac{u^{2}}{G^{2}}, \frac{u}{G})),
    \end{equation}
    where \(G = \max_{i}\norm{Y_{i}}_{\psi_{1}}\). \pause

    Since \(G \leq CK^{2} \leq C^{2}K^{2}\) and so \(G^{2} \leq C^{4}K^{4}\) we have
    \begin{equation}
      \label{eq:2}
      \Pr(\abs*{\frac{1}{n}\sum_{i=1}^{n}Y_{i}} > u) \leq 2 \exp(-(cn / C^{4}K^{4})\min(u^{2}, u)).
    \end{equation}
  \end{frame}

  \begin{frame}{Proof (Concentration of the \(\ell^{2}\) Norm)}
    We have shown that
    \begin{equation}
      \Pr(\abs*{\frac{1}{n}\norm{X}_{2}^{2} - 1} > u) \leq 2 \exp(-(cn / C^{4}K^{4})\min(u^{2}, u)).
    \end{equation}
  \end{frame}

  \begin{frame}{Proof (Concentration of the \(\ell^{2}\) Norm)}
    Note the following -- \pause
    \begin{itemize}
      \item for any \(z, \delta \geq 0\),
      \begin{equation}
        \label{eq:zp1-delta-observation}
        \abs{z - 1} \geq \delta \Rightarrow \abs{z^{2} - 1} \geq \max(\delta, \delta^{2}),
      \end{equation}
      \pause
      \item for any \(\delta \geq 0\),
      \begin{equation}
        \delta^{2} = \min(\max(\delta, \delta^{2}), \max(\delta, \delta^{2})^{2}).
      \end{equation}
    \end{itemize}
  \end{frame}

  \begin{frame}{Proof (Concentration of the \(\ell^{2}\) Norm)}
    By previous slide
    \begin{equation}
      \abs{\frac{1}{\sqrt{n}}\norm{X}_{2} - 1} \geq \delta \Rightarrow \abs{\frac{1}{n}\norm{X}^{2}_{2} - 1} \geq \max(\delta^{2}, \delta),
    \end{equation}
    \pause which means that
    \begin{equation}
      \Pr(\abs*{\frac{1}{\sqrt{n}}\norm{X}_{2} - 1} \geq \delta) \leq \Pr(\abs*{\frac{1}{n}\norm{X}^{2}_{2} - 1} \geq \max(\delta, \delta^{2})) \leq 2 \exp(-\frac{cn}{C^{4}K^{4}}\delta^{2}),
    \end{equation}
    where we have used that \(\delta^{2} = \min(\max(\delta, \delta^{2}), \max(\delta, \delta^{2})^{2})\).
  \end{frame}

  \begin{frame}{Proof (Concentration of the \(\ell^{2}\) Norm)}
    Letting \(t = \delta \sqrt{n}\) we obtain the bound
    \begin{equation}
      \Pr(\abs{\norm{X}_{2} - \sqrt{n}} \geq t) \leq 2\exp(-\frac{ct^{2}}{C^{4}K^{4}}),
    \end{equation}
    for any \(t \geq 0\) which is equal to what we wanted to show.
  \end{frame}

\end{document}
